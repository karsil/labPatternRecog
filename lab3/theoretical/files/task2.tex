\section*{Exercise T-3.4: Bayes Rule}

\subsection*{Problem}

An economics consulting firm has created a model to predict recessions. The model predicts a recession with probability 80\% when a recession is indeed coming and with probability 10\% when no recession is coming. The unconditional probability of falling into a recession is 20\%. If the model predicts a recession, what is the probability that a recession will indeed come?

\subsection*{Solution}
Define:
\begin{align}
	R_C &= \text{Recession coming}\nonumber \\
	P_R &= \text{Predicted recession}\nonumber
\end{align}
Given:
\begin{align}
	P(P_R|R_C) &= 0.8 \nonumber \\
	P(P_R|\sim R_C) &= 0.8 \nonumber \\
	P(R_C) &= 0.2 \nonumber
\end{align}
Find:
\begin{align}
	P(R_C|P_R) &= 0.8 \nonumber
\end{align}
Calculate the probability for a prediction $P(P_R)$:
\begin{align}
	P(P_R) &= 0.2 \cdot 0.8 + 0.8 \cdot 0.1 \nonumber \\
	&= 0.16 + 0.008 \nonumber \\
	&= 0.24 \nonumber 
\end{align}
Calculate $P(R_C|P_R)$:
\begin{align}
	P(R_C|P_R)) &= \frac{P(P_R|R_C) \cdot  P(R_C)}{P(P_R)} && \text{(Bayes Rule)} \nonumber \\
	&= \frac{0.8 \cdot 0.2}{0.24} \nonumber \\
	&= \frac{2}{3} \nonumber
\end{align}

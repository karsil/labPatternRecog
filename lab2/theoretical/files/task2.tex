\section*{Exercise T-2.2}

\subsection*{Problem}
The Census Bureau has estimated the following survival probabilities for men:
\begin{enumerate}
	\item probability that a man lives at least 70 years: 80 \%
	\item probability that a man lives at least 80 years: 50 \%
\end{enumerate}

What is the conditional probability that a man lives at least 80 years given that he has just celebrated his 70th birthday?

\subsection*{Solution}

Given probabilities:
\begin{align}
	p(70) =& 0.8: \text{(Chance a man lives at least 70 years)} \nonumber \\
	p(80) =& 0.5: \text{(Chance a man lives at least 80 years)} \nonumber
\end{align}

Calculation of $p(80|70)$:

\begin{align}
	p(70|80) =& 1 \text{ (no man can live 80 years if he died before living 70 years)} \nonumber \\
	p(80|70) =& \frac{p(70|80)\cdot p(80)}{p(70)} = \frac{1\cdot 0.5}{0.8} = 0.625 \nonumber
\end{align}


Solution: The conditional probability that a man lives at least 80 years given that he has just celebrated his 70th birthday is at 62.5\%.
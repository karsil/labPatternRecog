\section*{Exercise T-2.1}

\subsection*{Problem}
Consider a sample space $X$ comprising three possible outcomes $X = {v_1, v_2, v_3}$.\\ \\
We define the events
\begin{align}
	E = \{v_1,v_2\} \nonumber \\
	F = \{v_1,v_3\}\nonumber 
\end{align}
and denote by $E^c$ the complement of $E$.\\ \\
Compute $P(F|E^c)$, the conditional probability of $F$ given $E^c$ , using the conditional
probability formula for the events $A$ and $B$:\\

\begin{align}
P(A|B)=  \frac{P(A \cap B)}{P(B)} \nonumber 
\end{align}


\subsection*{Solution}

\begin{align}
	P(F|E^c) &= \frac{P(F\cap E^c)}{P(E^c)} \nonumber \\
	&= \frac{P(F) - P(F \cap E)}{1 - P(E)} \nonumber \\
	&= \frac{P( \{v_1\} \cup \{v_3\} ) - P(( \{v_1\} \cup \{v_3\} ) \cap (\{v_1\} \cup \{v_2\}))}{1 - P(\{v_1\} \cup \{v_2\})} \nonumber \\
	&= \frac{P( \{v_1\} \cup \{v_3\} ) - P(\{v_2\} \cap \{v_3\})}{1 - P(\{v_1\} \cup \{v_2\})} \nonumber \\
	&= \frac{(P(v_1) + P(v_3)) - P(v_2)\cdot P(v_3)}{1-(P(v_1)+P(v_2))}\nonumber
\end{align}
%Anderer Loesungsansatz:
%$ = \frac{P(E^c|F)\cdot P(F)}{P(E^c)}$
